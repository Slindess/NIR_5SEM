\documentclass[a4paper, 14pt, unknownkeysallowed]{extreport}
\usepackage[T2A]{fontenc}  % Поддержка кириллицы
\usepackage[utf8]{inputenc}  % Кодировка utf-8
\usepackage[russian]{babel}  % Поддержка русского языка
\usepackage{setspace}  % Пакет для изменения межстрочного интервала
\usepackage{geometry}  % Пакет для настройки полей страниц
\usepackage{amsmath}

\usepackage{titlesec}
\titleformat{\section}
	{\normalsize\bfseries}
	{\thesection}
	{1em}{}
\titlespacing*{\chapter}{1.5cm}{-22pt}{10pt}
\titlespacing*{\section}{1.5cm}{\baselineskip}{\baselineskip}
\titlespacing*{\subsection}{1.5cm}{\baselineskip}{\baselineskip}

\usepackage{titlesec}
\titleformat{\chapter}{\Large\bfseries}{\thechapter}{20pt}{\Large\bfseries}
\titleformat{\section}{\Large\bfseries}{\thesection}{20pt}{\Large\bfseries}

\newcommand{\chaptercentered}[1]{
	% Вот тут установить центрирование
	\titleformat{\chapter}{\Large\bfseries\centering}{\thechapter}{20pt}{\Large\bfseries}
    \chapter*{\MakeUppercase{#1}} % Центрирование заголовка
    \addcontentsline{toc}{chapter}{#1} % Добавление главы в оглавление
    \markboth{#1}{#1} % Обновление колонтитулов
    % Убрать центрирование
    \titleformat{\chapter}{\Large\bfseries}{\thechapter}{20pt}{\Large\bfseries}
}

% Настройка полей документа
\geometry{
  left=30mm,  % Левое поле
  right=10mm,  % Правое поле
  top=20mm,  % Верхнее поле
  bottom=20mm  % Нижнее поле
}

% Настройка шрифтов
\usepackage{fontspec}
\setmainfont{Times New Roman}
\onehalfspacing

\setlength{\parindent}{1.5cm}  % Отступ в начале абзаца
%\setlength{\parskip}{0em}

\sloppy

\frenchspacing
\usepackage{indentfirst}

\usepackage{fancyhdr}
\pagestyle{fancy}
\fancyhf{}  % Очищаем текущие настройки стиля
\fancyfoot[C]{\thepage}  % Установка нумерации по центру внизу страницы
%\thispagestyle{empty} - уберет нумерацию на конкретной странице (можно использовать на титульнике).
\renewcommand{\headrulewidth}{0pt}

\usepackage{caption}
\captionsetup{labelsep=endash}
\captionsetup[figure]{name={Рисунок}}
% Картинки
\usepackage{pgfplots}
\usetikzlibrary{datavisualization}
\usetikzlibrary{datavisualization.formats.functions}

\usepackage{graphicx}
\makeatletter
\def\@biblabel#1{#1. }
\makeatother

% Команды добавления изображений
\usepackage{float}
% Команда добавления изображений с параметром ширины
\newcommand{\img}[3][\textwidth]{ % Первый аргумент - ширина по умолчанию
    \begin{figure}[H]
    	\vspace{20pt}
        \centering
        \includegraphics[width=#1, height=200mm, keepaspectratio]{img/#2} % Используем переданную ширину
        \caption{#3}
        \label{img/#2}
    \end{figure}
}

\usepackage[justification=centering]{caption} % Настройка подписей float объектов


\usepackage{csvsimple}


\usepackage{svg}


% Переименуем
\renewcommand{\contentsname}{Содержание}
\titleformat{\bibname}[block]{\Large\bfseries}{}{0pt}{\Large\bfseries}
